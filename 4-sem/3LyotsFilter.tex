

\section{Фильтр Лио}

\subsection{Интерференционно-поляризационный фильтр Лио}

Интерференционно-поляризационный фильтр Вуда имеет спектр пропускания в виде узких полос,
разделенных промежутками такой же ширины. \\

Лио предложил монохроматический фильтр, состоящий из стопы фильтров Вуда, с толщинами
пластинок, увеличивающимися в геометрической прогрессии с показателем 2. При этом
выходной поляризатор первого фильтра служит входным поляризатором второго и т. д. См.
Рисунок 5.

\pic{0.8\linewidth}{LyotsFilter.jpg}{Интерференционно-поляризационный фильтр Лио}

Ширина полосы пропускания фильтра Лио определяется толщиной наиболее толстой пластинки
(формула \ref{WoodsFilterDelta}). Расстояние между полосами пропускания можно определить
по той же формуле, по толщине $ l $ самой тонкой пластинки. \\

Пропускание фильтра Лио можно вычислить как произведение пропусканий соответствующих
фильтров Вуда (формула \ref{WoodsFilterTransmission}):

\formula{LyotsFilterTransmission}{
    T = \prod_{k = 1}^n \cos^2 \frac{2^{k-1} \pi\mu l}{\lambda}
} \\

Здесть $ n $ -- число элементов, а $ l $ -- толщина наиболее тонкой пластинки. При этом
нежелательные полосы пропускания могут быть отфильтрованы с помощью абсорбционных или
интерференционных фильтров. \\

Интегральное пропускание фона у фильтра Лио составляет $ \approx 11\% $ от пропускания
в пределах полосы. Практически паразитный фон достигает $ 13 - 14\% $.

\subsection{Изменение длины волны пропускания}

Универсальность фильтра можно значительно повысить, если иметь возможность менять
длинну волны полосы пропускания. \\

В пределах $ 10 - 20 $ \AA \hspace{0.1cm} это можно делать изменением температуры фильтра. Более
радикальным решением задачи является ряд приемов, делающих фильтр управляемым в
широких пределах. Элементы фильтра делаются составными из двух клиньев, при смещении
одного из них толщина пластинки плавно меняется -- см. Рисунок 6. Есть и другие способы
сделать фильтр управляемым в более широких пределах -- можно использовать
электрооптические кристаллы, способные менять коэф-т двойного лучепреломления под
действием приложенного поля.

\pic{0.5\linewidth}{clinoidPlates.jpg}{
    Элементы фильтра Лио с пластинками, состоящими из двух клиньев
}

На рисунке 7 представлен внешний вид интерференционно-поляризационного фильтра Лио с
терморегулятором и его кривая пропускания. Фильтр имеет полуширину полосы пропускания
$ 0,3 $ \AA \hspace{0.1cm} и угловое поле $ 3^\circ $.

\pic{0.7\linewidth}{LyotsFilterSample.jpg}{
    Пример интерференционно-поляризационного фильтра Лио
}

\subsection{Области применения узкополосных фильтров}

Узкополосные фильтры (такие, как, например, фильтр Лио) применяются при проведении
специфичных фото-химических реакций и в области спектральных и фото-химических
исследований.
